\documentclass[kulak]{kulakarticle} % options: kulak (default) or kul

\usepackage[dutch]{babel}
\usepackage{siunitx}
\usepackage{mhchem}
\usepackage{graphicx}
\usepackage{flafter}
\usepackage{pdfpages}

\title{Literatuurstudie}
\author{Jelle Vanwijnsberghe}
\date{Academiejaar 2021 -- 2022}
\address{
	\textbf{Groep Wetenschap \& Technologie Kulak} \\
	Ingenieurswetenschappen \\
	Wetenschappelijk rekenen en schrijven}

\begin{document}

\maketitle

\section*{Het principe van $^1$$^4$\ce{C}-datering}

\section*{Inleiding}

Koolstof (\ce{C}) is een veel voorkomende stof op aarde. Het bevindt zich bijvoorbeeld in allerlei organisch materiaal, in bijna al onze brandstoffen en plastics, in diamanten, in potloden... Maar de meeste koolstof op aarde bevindt zich als koolstofdioxide in onze lucht. 

Er bestaan twee stabiele koolstofisotopen: $^1$$^2$\ce{C} ($\sim$\num{99}\verb|%|) en $^1$$^3$\ce{C} ($\sim$\num{1}\verb|%|), maar daarnaast bestaan er nog een heleboel instabiele varianten, waarvan $^1$$^4$\ce{C} het bekendste is, radioactief maar ongevaarlijk. Met behulp van dit isotoop en zijn radioactieve eigenschap kan de ouderdom van organisch materiaal bepaald worden, daarom noemen we dit $^1$$^4$\ce{C}-datering.

Deze dateringsmethode kan enkel worden toegepast op biologische materialen (zoals botten, vlees, hout…) die niet ouder zijn dan \num{60000} jaar. 

\end{document}